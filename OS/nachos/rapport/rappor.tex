%%%%%%%%%%%%%%%%%%%%%%%%%%%%%%%%%%%%%%%%%
%
%%%%%%%%%%%%%%%%%%%%%%%%%%%%%%%%%%%%%%%%%

%----------------------------------------------------------------------------------------
%	PACKAGES AND OTHER DOCUMENT CONFIGURATIONS
%----------------------------------------------------------------------------------------

\documentclass{article}

\usepackage{fancyhdr} % Required for custom headers
\usepackage{lastpage} % Required to determine the last page for the footer
\usepackage{extramarks} % Required for headers and footers
\usepackage[usenames,dvipsnames]{color} % Required for custom colors
\usepackage{graphicx} % Required to insert images
\usepackage{listings} % Required for insertion of code
\usepackage{courier} % Required for the courier font
\usepackage{lipsum} % Used for inserting dummy 'Lorem ipsum' text into the template
\usepackage[utf8]{inputenc}
\usepackage[T1]{fontenc}
\usepackage{lmodern}
\usepackage{verbatim}



% Margins
\topmargin=-0.45in
\evensidemargin=0in
\oddsidemargin=0in
\textwidth=6.5in
\textheight=9.0in
\headsep=0.25in

\linespread{1.1} % Line spacing

% Set up the header and footer
\pagestyle{fancy}
\lhead{\hmwkAuthorName} % Top left header
\chead{\hmwkClass} % Top center head
\rhead{\firstxmark} % Top right header
\lfoot{\lastxmark} % Bottom left footer
\cfoot{} % Bottom center footer
\rfoot{Page\ \thepage\ of\ \protect\pageref{LastPage}} % Bottom right footer
\renewcommand\headrulewidth{0.4pt} % Size of the header rule
\renewcommand\footrulewidth{0.4pt} % Size of the footer rule

\setlength\parindent{0pt} % Removes all indentation from paragraphs


% code listing settings
\usepackage{listings}
\lstset{
    language=Python,
    basicstyle=\ttfamily\small,
    aboveskip={1.0\baselineskip},
    belowskip={1.0\baselineskip},
    columns=fixed,
    extendedchars=true,
    breaklines=true,
    tabsize=4,
    prebreak=\raisebox{0ex}[0ex][0ex]{\ensuremath{\hookleftarrow}},
    frame=lines,
    showtabs=false,
    showspaces=false,
    showstringspaces=false,
    keywordstyle=\color[rgb]{0.627,0.126,0.941},
    commentstyle=\color[rgb]{0.133,0.545,0.133},
    stringstyle=\color[rgb]{01,0,0},
    numbers=left,
    numberstyle=\small,
    stepnumber=1,
    numbersep=10pt,
    captionpos=t,
    escapeinside={\%*}{*)}
}
%----------------------------------------------------------------------------------------
%	DOCUMENT STRUCTURE COMMANDS
%	Skip this unless you know what you're doing
%----------------------------------------------------------------------------------------

% Header and footer for when a page split occurs within a problem environment
\newcommand{\enterProblemHeader}[1]{\nobreak\extramarks{#1}{#1 continued on next page\ldots}\nobreak\nobreak\extramarks{#1 (continued)}{#1 continued on next page\ldots}\nobreak}

% Header and footer for when a page split occurs between problem environments
\newcommand{\exitProblemHeader}[1]{\nobreak\extramarks{#1 (continued)}{#1 continued on next page\ldots}\nobreak\nobreak\extramarks{#1}{}\nobreak}

\setcounter{secnumdepth}{0} % Removes default section numbers
\newcounter{homeworkProblemCounter} % Creates a counter to keep track of the number of problems

\newcommand{\homeworkProblemName}{}
\newenvironment{homeworkProblem}[1][Problem \arabic{homeworkProblemCounter}]{ % Makes a new environment called homeworkProblem which takes 1 argument (custom name) but the default is "Problem #"
\stepcounter{homeworkProblemCounter} % Increase counter for number of problems
\renewcommand{\homeworkProblemName}{#1} % Assign \homeworkProblemName the name of the problem
\section{\homeworkProblemName} % Make a section in the document with the custom problem count
\enterProblemHeader{\homeworkProblemName} % Header and footer within the environment
}
{\exitProblemHeader{\homeworkProblemName} % Header and footer after the environment
}

\newcommand{\problemAnswer}[1]{\noindent\framebox[\columnwidth][c]{\begin{minipage}{0.98\columnwidth}#1\end{minipage}} % Makes the box around the problem answer and puts the content inside
}

\newcommand{\homeworkSectionName}{}
\newenvironment{homeworkSection}[1]{ % New environment for sections within homework problems, takes 1 argument - the name of the section
\renewcommand{\homeworkSectionName}{#1} % Assign \homeworkSectionName to the name of the section from the environment argument
\subsection{\homeworkSectionName} % Make a subsection with the custom name of the subsection
\enterProblemHeader{\homeworkProblemName\ [\homeworkSectionName]} % Header and footer within the environment
}{
\enterProblemHeader{\homeworkProblemName} % Header and footer after the environment
}

%----------------------------------------------------------------------------------------
%	NAME AND CLASS SECTION
%----------------------------------------------------------------------------------------

\newcommand{\hmwkTitle}{NachOS project} % Assignment title
\newcommand{\hmwkDueDate}{Monday,\ January\ 1,\ 2012} % Due date
\newcommand{\hmwkClass}{J1INAW11} % Course/class
\newcommand{\hmwkClassInstructor}{NAMYST Raymond} % Teacher/lecturer
\newcommand{\hmwkAuthorName}{VER VALEM Willian} % Your name

%----------------------------------------------------------------------------------------
%	TITLE PAGE
%----------------------------------------------------------------------------------------

\title{\vspace{2in}
\textmd{\textbf{\hmwkClass:\ \hmwkTitle}}\\
\normalsize\vspace{0.1in}\small{Due\ on\ \hmwkDueDate}\\
\vspace{0.1in}\large{\textit{\hmwkClassInstructor}}
\vspace{3in}
}

\author{\textbf{\hmwkAuthorName}}
\date{} % Insert date here if you want it to appear below your name

%----------------------------------------------------------------------------------------

\begin{document}

\maketitle

%----------------------------------------------------------------------------------------
%	TABLE OF CONTENTS
%----------------------------------------------------------------------------------------

%\setcounter{tocdepth}{1} % Uncomment this line if you don't want subsections listed in the ToC

\newpage
\tableofcontents
\newpage

%----------------------------------------------------------------------------------------
%	PROBLEM 1
%----------------------------------------------------------------------------------------

% To have just one problem per page, simply put a \clearpage after each problem

\begin{homeworkProblem}[Introduction]
\end{homeworkProblem}
\begin{homeworkProblem}[Points délicats]
    \begin{homeworkSection}{Entrée-sorties synchrones}
        the mistake done in this part of the project was to not include a break in case of end of file, end of string or new line
        on the function SynchGetString witch caused the program to leak the previus string into the new one.
        
        that was fixed by including the lines 4 to 6:
        \begin{lstlisting}[label={list:first},caption=SynchGetString.]
            for(int i = 0;i<n;i++){
                readAvail->P();
                s[i]=console->GetChar();
                if(s[i] == EOF || s[i]=='\0' || s[i] == '\n'){
                    break;
                }
            }
        \end{lstlisting}
    \end{homeworkSection}
    \begin{homeworkSection}{Appel système Putchar}
        following the steps given by the exercise description was enought to implement these syscall    
    \end{homeworkSection}
    \begin{homeworkSection}{Appel système PutString}
        for this exercise witch was the most complicated on my humble opinion 
        i got really stuck on the part to manipulate strings via a small buffer witch caused some problems.
        for the solution of this problem i declared 2 constants BUFFER\_SIZE and MAXI\_SIZE. \newline

        then reading a sequence of charateres of BUFFER\_SIZE or until we find a '\textbackslash0' 
        and if a BUFFER\_SIZE was readed but no '\textbackslash0' was readed we repeat the process until a it finds a '\textbackslash0'
        or a MAXI\_SIZE was reached in case a MAXI\_SIZE is reached with no '\textbackslash0' the '\textbackslash0' is added to the and of the string. \newline

        for the function copyStringFromMachine i decided to place it at the file machine.h e implement in machine.c as the use of this function looks really tight to the other functions form machine
    \end{homeworkSection}
    \begin{homeworkSection}{Appel système Exit}
        for this syscall to implement the asked on the exercise was necessaire to move the machine halt to the syscall EXIT
        witch works well in case of a unic process. \newline

        after some time trying to figure out how to handle this problem i was instructed by the teacher to leave it as it is 
        because the part were we have to track the process and implement the proper exit syscall will be done one the next step of the project.
            \newpage
        so the exit syscal was simply implemented like this:

        \begin{lstlisting}[label={list:first},caption=SC\_Exit.]
                            interrupt->Halt();
                            break;
        \end{lstlisting}
    
        as for the main value return it cam be recoverd by reading the register 4
        after change the  file start.S to store the register 2 on register 4 instead of register 0 like:
        \begin{lstlisting}[label={list:first},caption=\_\_start at start.S]
                .globl __start
                .ent	__start
            __start:
                jal	main
                move	$4,$2		
                jal	Exit	 
                .end __start
        \end{lstlisting}
        
        so the value of the register is stored at register 4 but up to now i don't really know if i should right this value back to register 2 as a return of the Exit syscall
        
    \end{homeworkSection}
    \begin{homeworkSection}{Appel système GetChar}
        was enought to call SynchGetChar and write the return value in the register 2
    \end{homeworkSection}
    \begin{homeworkSection}{Appel système GetString}
        inspired by the implementatio of PutString it first calls the SynchGetString then write it on the adrees of the user space with the call of the function copyStringToMachine
        and remenbering to always insert a '\textbackslash0' at the end 
    \end{homeworkSection}
    \begin{homeworkSection}{Bonus printf, PutInt, GetInt}
        Not implemented as the the duo of the project was sick and i had no time to do all 
        but will be implemented for the sake off learning
    \end{homeworkSection}

    \begin{homeworkSection}{tests}
        for the test i did implement some extra functions inside putchar.c 
        for testing the GetString and PutString i implemented a loop that works like the consoleTest on userprog
        reading a string and printing it back that offer the ability to test diferent lenght of string for both syslcalls 
    \end{homeworkSection}
\end{homeworkProblem}
%----------------------------------------------------------------------------------------
\end{document}

