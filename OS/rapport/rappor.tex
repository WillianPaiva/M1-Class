%%%%%%%%%%%%%%%%%%%%%%%%%%%%%%%%%%%%%%%%%
%
%%%%%%%%%%%%%%%%%%%%%%%%%%%%%%%%%%%%%%%%%

%----------------------------------------------------------------------------------------
%   PACKAGES AND OTHER DOCUMENT CONFIGURATIONS
%----------------------------------------------------------------------------------------

\documentclass{article}

\usepackage{fancyhdr} % Required for custom headers
\usepackage{lastpage} % Required to determine the last page for the footer
\usepackage{extramarks} % Required for headers and footers
\usepackage[usenames,dvipsnames]{color} % Required for custom colors
\usepackage{graphicx} % Required to insert images
\usepackage{listings} % Required for insertion of code
\usepackage{courier} % Required for the courier font
\usepackage{lipsum} % Used for inserting dummy 'Lorem ipsum' text into the template
\usepackage[utf8]{inputenc}
\usepackage[T1]{fontenc}
\usepackage{lmodern}
\usepackage{verbatim}



% Margins
\topmargin=-0.45in
\evensidemargin=0in
\oddsidemargin=0in
\textwidth=6.5in
\textheight=9.0in
\headsep=0.25in

\linespread{1.1} % Line spacing

% Set up the header and footer
\pagestyle{fancy}
\lhead{\hmwkAuthorName} % Top left header
\chead{\hmwkClass} % Top center head
\rhead{\firstxmark} % Top right header
\lfoot{\lastxmark} % Bottom left footer
\cfoot{} % Bottom center footer
\rfoot{Page\ \thepage\ of\ \protect\pageref{LastPage}} % Bottom right footer
\renewcommand\headrulewidth{0.4pt} % Size of the header rule
\renewcommand\footrulewidth{0.4pt} % Size of the footer rule

\setlength\parindent{0pt} % Removes all indentation from paragraphs


% code listing settings
\usepackage{listings}
\lstset{
    language=Python,
    basicstyle=\ttfamily\small,
    aboveskip={1.0\baselineskip},
    belowskip={1.0\baselineskip},
    columns=fixed,
    extendedchars=true,
    breaklines=true,
    tabsize=4,
    prebreak=\raisebox{0ex}[0ex][0ex]{\ensuremath{\hookleftarrow}},
    frame=lines,
    showtabs=false,
    showspaces=false,
    showstringspaces=false,
    keywordstyle=\color[rgb]{0.627,0.126,0.941},
    commentstyle=\color[rgb]{0.133,0.545,0.133},
    stringstyle=\color[rgb]{01,0,0},
    numbers=left,
    numberstyle=\small,
    stepnumber=1,
    numbersep=10pt,
    captionpos=t,
    escapeinside={\%*}{*)}
}
%----------------------------------------------------------------------------------------
%   DOCUMENT STRUCTURE COMMANDS
%   Skip this unless you know what you're doing
%----------------------------------------------------------------------------------------

% Header and footer for when a page split occurs within a problem environment
\newcommand{\enterProblemHeader}[1]{\nobreak\extramarks{#1}{#1 continued on next page\ldots}\nobreak\nobreak\extramarks{#1 (continued)}{#1 continued on next page\ldots}\nobreak}

% Header and footer for when a page split occurs between problem environments
\newcommand{\exitProblemHeader}[1]{\nobreak\extramarks{#1 (continued)}{#1 continued on next page\ldots}\nobreak\nobreak\extramarks{#1}{}\nobreak}

\setcounter{secnumdepth}{0} % Removes default section numbers
\newcounter{homeworkProblemCounter} % Creates a counter to keep track of the number of problems

\newcommand{\homeworkProblemName}{}
\newenvironment{homeworkProblem}[1][Problem \arabic{homeworkProblemCounter}]{ % Makes a new environment called homeworkProblem which takes 1 argument (custom name) but the default is "Problem #"
\stepcounter{homeworkProblemCounter} % Increase counter for number of problems
\renewcommand{\homeworkProblemName}{#1} % Assign \homeworkProblemName the name of the problem
\section{\homeworkProblemName} % Make a section in the document with the custom problem count
\enterProblemHeader{\homeworkProblemName} % Header and footer within the environment
}
{\exitProblemHeader{\homeworkProblemName} % Header and footer after the environment
}

\newcommand{\problemAnswer}[1]{\noindent\framebox[\columnwidth][c]{\begin{minipage}{0.98\columnwidth}#1\end{minipage}} % Makes the box around the problem answer and puts the content inside
}

\newcommand{\homeworkSectionName}{}
\newenvironment{homeworkSection}[1]{ % New environment for sections within homework problems, takes 1 argument - the name of the section
\renewcommand{\homeworkSectionName}{#1} % Assign \homeworkSectionName to the name of the section from the environment argument
\subsection{\homeworkSectionName} % Make a subsection with the custom name of the subsection
\enterProblemHeader{\homeworkProblemName\ [\homeworkSectionName]} % Header and footer within the environment
}{
\enterProblemHeader{\homeworkProblemName} % Header and footer after the environment
}

%----------------------------------------------------------------------------------------
%   NAME AND CLASS SECTION
%----------------------------------------------------------------------------------------

\newcommand{\hmwkTitle}{NachOS project} % Assignment title
\newcommand{\hmwkDueDate}{Monday,\ November\ 9,\ 2015} % Due date
\newcommand{\hmwkClass}{J1INAW11} % Course/class
\newcommand{\hmwkClassInstructor}{NAMYST Raymond} % Teacher/lecturer
\newcommand{\hmwkAuthorName}{VER VALEM Willian , RAKOTONIERA Hoby} % Your name

%----------------------------------------------------------------------------------------
%   TITLE PAGE
%----------------------------------------------------------------------------------------

\title{\vspace{2in}
\textmd{\textbf{\hmwkClass:\ \hmwkTitle}}\\
\normalsize\vspace{0.1in}\small{Due\ on\ \hmwkDueDate}\\
\vspace{0.1in}\large{\textit{\hmwkClassInstructor}}
\vspace{3in}
}

\author{\textbf{\hmwkAuthorName}}
\date{} % Insert date here if you want it to appear below your name

%----------------------------------------------------------------------------------------

\begin{document}

\maketitle

%----------------------------------------------------------------------------------------
%   TABLE OF CONTENTS
%----------------------------------------------------------------------------------------

%\setcounter{tocdepth}{1} % Uncomment this line if you don't want subsections listed in the ToC

\newpage
\tableofcontents
\newpage

%----------------------------------------------------------------------------------------
%   PROBLEM 1
%----------------------------------------------------------------------------------------

% To have just one problem per page, simply put a \clearpage after each problem

\section{Introduction}
Pour cette partie du projet, il était demandé de mettre en place un niveau utilisateur Multithreading avec NachOS.
\newline
Pour réaliser ce travail, on a une suivi une série d'étapes fournies par l'enseignant, en commençant tout d'abord avec la possibilité de créer un seul thread d'une façon rudimentaire, et en l'améliorant ensuite, jusqu'à arriver à la possibilité de faire fonctionner un nombre indéterminé de threads, et synchroniser la façon dont ils se terminent.

L'objectif était de comprendre comment les threads sont créés, gérés, chargés et terminés, c'est à dire le cycle de vie complet d'un thread; et ensuite de mettre en oeuvre ce cycle de vie dans NachOS.

\section{Multithreading}
\subsection{Multithreading dans les programmes utilisateurs}

\subsubsection{ThreadCreat}
La mise en oeuvre de l'appel système "\textit{ThreadCreat}" a été réalisée en créant une fonction qui a la responsabilité de mémoriser le "pointeur" de la fonction et le pointeur des arguments, de créer un thread et de le démarrer en utilisant \textit{StartUserThread} et en lui passant les arguments. 

\subsubsection{ThreadExit}
Dans la première partie du projet, \textit{ThreadExit} avait à effectuer une tâche simple, qui était d'appeler \textit{cutrrentThread->Finish()}.

\subsubsection{StartUserThread}
StartUserThread était le point clé de la création d'un thread, puisqu'il est responsable de charger le thread en mémoire et de le faire tourner. Pour réaliser cela, une nouvelle fonction similaire à \textit{InitRegisters} a été mise en oeuvre dans la classe \textit{addrespace}, la fonction \textit{InitUserRegisters} qui charge les registres et pointe vers eux (au sommet de la pile) - 256.

Voici le code snippet de la fonction :
\begin{verbatim}

void
AddrSpace::InitUserRegisters (int f,int arg)
{
    printf("inside\n");
    int threadNumber = currentThread->threadNumber;
    //int i;
    //for (i = 0; i < NumTotalRegs; i++)
    //machine->WriteRegister (i, 0);
    // Initial program counter -- must be location of "Start"
    machine->WriteRegister (PCReg, f);
    machine->WriteRegister(4, arg);
    machine->WriteRegister (NextPCReg, machine->ReadRegister(PCReg) + 4);
    machine->WriteRegister (StackReg, numPages * PageSize -( 256 * threadNumber)-16);
    DEBUG ('a', "Initializing USER stack register to 0x%x\n", numPages * PageSize - 256);
}
\end{verbatim}

Il utilise une variable globale \textit{threadNumber} pour allouer l'espace approprié sur la pile, et qui, à ce stade du projet, avait pour seule valeur "1".


\subsection{Plusieurs threads par processus}

\subsubsection{ThreadCreat}

Lorsque nous avons eu besoin de gérer plus d'un thread par processus, il a fallu changer cette fonction, car elle a créé des problèmes. 
Ces problèmes sont les suivants : 
\begin{itemize}
\item{ThreadCreate avait besoin de répondre s'il était possible de créer le thread}
\item{il ne pouvait pas créer un thread s'il ne savait pas s'il y avait des slots libres}
\item{une fois qu'un thread était créé, il devait occuper ce slot pour éviter que d'autres threads y accèdent.}


\end{itemize}

Pour résoudre ces problèmes, l'approche choisie a été d'accéder au BitMap du processus, et de donner le nombre de threads avec cette fonction, et de cette manière, une fois qu'un thread est créé, il a déjà son numéro et le slot sur lequel il sera chargé ; cela évite ainsi que le thread principal se termine avant que le thread soit exécuté. 
\newline

Comme cette méthode incrémente également le \textit{threadCounter}, une fois qu'un \textit{StartUserThread} est appelé, le thread a déjà son espace affecté, auquel il est alloué. Cette démarche a été réalisée comme suit : 
\begin{verbatim}
int do_ThreadCreate(int f, int arg){
    int* temp = new int[2];
    temp[0] = f;
    temp[1] = arg;
    if(currentThread->space->threadBitMap->NumClear() > 0){
        Thread *t = new Thread ("new thread");
        t->threadNumber = currentThread->space->threadBitMap->Find();
        currentThread->space->threadCounter++;
        t->Start(StartUserThread,temp);
        return 0;
    }else{
        return -1;
    }
}
\end{verbatim}

\subsubsection{ThreadExit}

Comme maintenant il a un nombre indéterminé de threads, ThreadExit doit aussi garder la trace des threads et des slots et pour effectuer ce comportement, il appelle \textit{currentThread->Finish()}, libère le \textit{BitMap} et réduit le \textit{threadCounter}.

De cette manière, NachOS peut garder la trace des threads actifs et réutiliser un espace pour allouer à de nouveaux threads et dans la classe  \textit{exception}, le  \textit{sysxall SC\_Exit} et le \textit{SC\_ThreadExit} ont été modifiés et font tous les deux la même chose. 
\begin{verbatim}
    if(currentThread->space->threadCounter == 1){
        interrupt->Halt ();
    }else{
        do_ThreadExit();
    }
    break;
\end{verbatim}

Ainsi, avec cette méthode, ce n'est que le dernier thread qui est capable d'appeler le \textit{interrupt->Halt()}.


\subsubsection{SynchConsole and Lock}
Pour pouvoir utiliser \textit{SynchConsole} avec les threads, un lock a été mis en place sur la classe \textit{Synch}. Ce qu'il fait, essentiellement, est de créer un \textit{semaphore} initialisé à un et arrêter le system interrupt à \textit{aquire} et redémarrer le system interrupt à \textit{release}

Ainsi, grâce qu fait que j'aie mis ce système en place, j'ai été capable de vérouiller l'appel de \textit{SynchPutString} et \textit{SynchGetString} et de cette manière, un thread est capable d'imprimer et utiliser ces appels sans interruptions, mais lorsque ces fonctions sont appelées par le syscall à \textit{exception} il peut être interrompu par le système à la fin du buffer. Le meilleur moyen d'exécuter un résultat à ce niveau est de verrouiller cette fonction au niveau du syscall, à \textit{exception}. A ce moment là, il n'est pas mis en oeuvre, donc quand des chaînes longues (plus longues que le buffer) dépendant de \textit{-rs} passés à NachOS, la chaîne peut être coupée au milieu. 

\section{tests}

Pour tester les threads, j'ai utilisé un simple bout de code, comme suit : 

\begin{verbatim}

void printString(char* c, int x)
{
    c[6]=48 + counter++;
    PutString(c);
    ThreadExit();
}

int main()
{

    char* stt = "\n\n\nth=0\n\n\n";
    char* mai = "\n\n\nmain\n\n\n";
    int x;
    for(x=0;x<4;x++){
        ThreadCreate(printString,stt);
    }
    PutString(mai);
    for(x=0;x<100;x++){}
    for(x=0;x<4;x++){
        ThreadCreate(printString,stt);
    } 
    return 0;
}

\end{verbatim}

De cette façon, j'ai été capable de tester où le programme crée les threads, quand il les exécute et les libère correctement. 
Le problème que j'ai eu à résoudre dépend du \textit{-rs} passé à NachOS. Le second cycle n'est pas exécuté tant que les 4 premiers threads ne sont pas finis, donc ça dépend comment le scheduler priorise l'exécution des threads. 
%----------------------------------------------------------------------------------------
\end{document}


