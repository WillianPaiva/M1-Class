%%%%%%%%%%%%%%%%%%%%%%%%%%%%%%%%%%%%%%%%%
%
%%%%%%%%%%%%%%%%%%%%%%%%%%%%%%%%%%%%%%%%%

%----------------------------------------------------------------------------------------
%   PACKAGES AND OTHER DOCUMENT CONFIGURATIONS
%----------------------------------------------------------------------------------------

\documentclass{article}

\usepackage{fancyhdr} % Required for custom headers
\usepackage{lastpage} % Required to determine the last page for the footer
\usepackage{extramarks} % Required for headers and footers
\usepackage[usenames,dvipsnames]{color} % Required for custom colors
\usepackage{graphicx} % Required to insert images
\usepackage{listings} % Required for insertion of code
\usepackage{courier} % Required for the courier font
\usepackage{lipsum} % Used for inserting dummy 'Lorem ipsum' text into the template
\usepackage[utf8]{inputenc}
\usepackage[T1]{fontenc}
\usepackage{lmodern}
\usepackage{verbatim}



% Margins
\topmargin=-0.45in
\evensidemargin=0in
\oddsidemargin=0in
\textwidth=6.5in
\textheight=9.0in
\headsep=0.25in

\linespread{1.1} % Line spacing

% Set up the header and footer
\pagestyle{fancy}
\lhead{\hmwkAuthorName} % Top left header
\chead{\hmwkClass} % Top center head
\rhead{\firstxmark} % Top right header
\lfoot{\lastxmark} % Bottom left footer
\cfoot{} % Bottom center footer
\rfoot{Page\ \thepage\ of\ \protect\pageref{LastPage}} % Bottom right footer
\renewcommand\headrulewidth{0.4pt} % Size of the header rule
\renewcommand\footrulewidth{0.4pt} % Size of the footer rule

\setlength\parindent{0pt} % Removes all indentation from paragraphs


% code listing settings
\usepackage{listings}
\lstset{
    language=Python,
    basicstyle=\ttfamily\small,
    aboveskip={1.0\baselineskip},
    belowskip={1.0\baselineskip},
    columns=fixed,
    extendedchars=true,
    breaklines=true,
    tabsize=4,
    prebreak=\raisebox{0ex}[0ex][0ex]{\ensuremath{\hookleftarrow}},
    frame=lines,
    showtabs=false,
    showspaces=false,
    showstringspaces=false,
    keywordstyle=\color[rgb]{0.627,0.126,0.941},
    commentstyle=\color[rgb]{0.133,0.545,0.133},
    stringstyle=\color[rgb]{01,0,0},
    numbers=left,
    numberstyle=\small,
    stepnumber=1,
    numbersep=10pt,
    captionpos=t,
    escapeinside={\%*}{*)}
}
%----------------------------------------------------------------------------------------
%   DOCUMENT STRUCTURE COMMANDS
%   Skip this unless you know what you're doing
%----------------------------------------------------------------------------------------

% Header and footer for when a page split occurs within a problem environment
\newcommand{\enterProblemHeader}[1]{\nobreak\extramarks{#1}{#1 continued on next page\ldots}\nobreak\nobreak\extramarks{#1 (continued)}{#1 continued on next page\ldots}\nobreak}

% Header and footer for when a page split occurs between problem environments
\newcommand{\exitProblemHeader}[1]{\nobreak\extramarks{#1 (continued)}{#1 continued on next page\ldots}\nobreak\nobreak\extramarks{#1}{}\nobreak}

\setcounter{secnumdepth}{0} % Removes default section numbers
\newcounter{homeworkProblemCounter} % Creates a counter to keep track of the number of problems

\newcommand{\homeworkProblemName}{}
\newenvironment{homeworkProblem}[1][Problem \arabic{homeworkProblemCounter}]{ % Makes a new environment called homeworkProblem which takes 1 argument (custom name) but the default is "Problem #"
\stepcounter{homeworkProblemCounter} % Increase counter for number of problems
\renewcommand{\homeworkProblemName}{#1} % Assign \homeworkProblemName the name of the problem
\section{\homeworkProblemName} % Make a section in the document with the custom problem count
\enterProblemHeader{\homeworkProblemName} % Header and footer within the environment
}
{\exitProblemHeader{\homeworkProblemName} % Header and footer after the environment
}

\newcommand{\problemAnswer}[1]{\noindent\framebox[\columnwidth][c]{\begin{minipage}{0.98\columnwidth}#1\end{minipage}} % Makes the box around the problem answer and puts the content inside
}

\newcommand{\homeworkSectionName}{}
\newenvironment{homeworkSection}[1]{ % New environment for sections within homework problems, takes 1 argument - the name of the section
\renewcommand{\homeworkSectionName}{#1} % Assign \homeworkSectionName to the name of the section from the environment argument
\subsection{\homeworkSectionName} % Make a subsection with the custom name of the subsection
\enterProblemHeader{\homeworkProblemName\ [\homeworkSectionName]} % Header and footer within the environment
}{
\enterProblemHeader{\homeworkProblemName} % Header and footer after the environment
}

%----------------------------------------------------------------------------------------
%   NAME AND CLASS SECTION
%----------------------------------------------------------------------------------------

\newcommand{\hmwkTitle}{NachOS project} % Assignment title
\newcommand{\hmwkDueDate}{Monday,\ October\ 12,\ 2015} % Due date
\newcommand{\hmwkClass}{J1INAW11} % Course/class
\newcommand{\hmwkClassInstructor}{NAMYST Raymond} % Teacher/lecturer
\newcommand{\hmwkAuthorName}{VER VALEM Willian , RAKOTONIERA Hoby} % Your name

%----------------------------------------------------------------------------------------
%   TITLE PAGE
%----------------------------------------------------------------------------------------

\title{\vspace{2in}
\textmd{\textbf{\hmwkClass:\ \hmwkTitle}}\\
\normalsize\vspace{0.1in}\small{Due\ on\ \hmwkDueDate}\\
\vspace{0.1in}\large{\textit{\hmwkClassInstructor}}
\vspace{3in}
}

\author{\textbf{\hmwkAuthorName}}
\date{} % Insert date here if you want it to appear below your name

%----------------------------------------------------------------------------------------

\begin{document}

\maketitle

%----------------------------------------------------------------------------------------
%   TABLE OF CONTENTS
%----------------------------------------------------------------------------------------

%\setcounter{tocdepth}{1} % Uncomment this line if you don't want subsections listed in the ToC

\newpage
\tableofcontents
\newpage

%----------------------------------------------------------------------------------------
%   PROBLEM 1
%----------------------------------------------------------------------------------------

% To have just one problem per page, simply put a \clearpage after each problem

\begin{homeworkProblem}[Introduction]
J'ai réalisé l'ensemble des appels systèmes qui étaient demandés : \textit{PutChar}, \textit{PutString},\textit{Exit}, \textit{GetChar}, \textit{GetString}. L'ensemble du projet a été réalisé sans difficultés particulières, hormis des difficultés de temps. C'est pour cette raison que je n'ai pas pu réaliser les bonus \textit{printf}, \textit{PutInt} et \textit{GetInt}. 
\newline

J'ai passé un certain temps, pour commencer, à essayer de comprendre le code, et de me faire une idée du fonctionnement du programme. Cela a été ma principale difficulté. \newline


Après ce temps d'adaptation et de compréhension, j'ai pu terminer la totalité de la mise en oeuvre de ces exercices. La seule chose que je n'ai pas pu finir c'est, pour la raison de temps évoquée plus haut, celle des bonus.
\newline


Pour ce qui a été fait, tout fonctionne correctement. Après avoir testé les fonctionnements de ces appels système, je n'ai pas pu trouver de bug. J'espère simplement que ce n'est pas par manque de parce que je n'ai pas su les détecter\ldots
\newline


En ce qui concerne les choses que je n'ai pas eu le temps de réaliser, je suppose qu'elles seraient du même ordre de difficulté que ce que j'ai déjà fait, et donc il me semble que je devrais y arriver de même, en espérant là aussi que je serais capable de trouver d'éventuels bugs.
\newline

Je pense qu'il me faudrait deux jours de travail de plus pour finir le bonus, à condition que je puisse m'y consacrer exclusivement. 



\end{homeworkProblem}
\begin{homeworkProblem}[Points délicats]
    \begin{homeworkSection}{Entrée-sorties synchrones}
    L'erreur commise dans cette partie du projet a été de ne pas inclure de \textit{break} en cas de fin de fichier (\textit{EOF}), fin de chaîne de caractères ou de nouvelle ligne au sein de la fonction \textit{SynchGetString} ce qui a amené le programme à faire fuir la chaîne précédente dans la nouvelle.\newline    
   
   Cette erreur a été réparée en insérant les lignes 4 à 6 :
        
        \begin{lstlisting}[label={list:first},caption=SynchGetString.]
            for(int i = 0;i<n;i++){
                readAvail->P();
                s[i]=console->GetChar();
                if(s[i] == EOF || s[i]=='\0' || s[i] == '\n'){
                    break;
                }
            }
        \end{lstlisting}
    \end{homeworkSection}
    \begin{homeworkSection}{Appel système Putchar}
        Suivre les étapes fournies dans la description de l'exercice a été suffisant pour mettre en place ces \textit{syscall}
        
    \end{homeworkSection}
    \begin{homeworkSection}{Appel système PutString}
        Pour cet exercice, qui à mon avis était le plus compliqué, j'ai vraiment été bloqué au niveau de la partie qui consistait à manipuler des chaînes de caractères via un petit buffer, ce qui a posé quelques problèmes. \newline
        
        Afin de trouver une solution à ce problème, j'ai déclaré deux constantes : \textit{BUFFER\_SIZE} et \textit{MAXI\_SIZE}.\newline
        
        De ce fait, lorsqu'on lit une séquence de caractères dans \textit{BUFFER\_SIZE} ou jusqu'à ce qu'on trouve une commande '\textbackslash0', et si un \textit{BUFFER\_SIZE} a été lu mais qu'il n'y avait pas de '\textbackslash0', le programme répète la séquence jusqu'à ce qu'il trouve un '\textbackslash0'. Ou si on lit un \textit{MAXI\_SIZE}, au cas où un \textit{MAXI\_SIZE} était lu mais sans '\textbackslash0', le '\textbackslash0' est ajouté à la fin de la chaîne.\newline
        
        Pour la fonction copyStringFromMachine, j'ai décidé de la placer dans le fichier \textit{machine.h implement in machine.c} car l'usage de cette fonction semble étroitement relié aux autres fonctions de la machine.       
        
     
    \end{homeworkSection}
    \begin{homeworkSection}{Appel système Exit}
    
    Pour mettre en oeuvre cet appel système comme il est demandé dans l'exercice, il a fallu copier l'appel à fonction \textit{halt} dans l'appel système \textit{sc\_Exit}
   ce qui fonctionne bien dans le cas d'un process unique.  \newline
   
   Après avoir passé quelque temps à essayer d'imaginer comment gérer ce problème, j'ai reçu le conseil de l'enseignant qui m'a dit de laisser les choses en l'état, parce que la partie dans laquelle nous devons contrôler la quantité de process et mettre en oeuvre l'appel système approprié sera traité dans l'étape suivante du projet. 
   
       Donc, l'exit syscal a simplement été mis en oeuvre de la façon suivante : 
    
    \begin{lstlisting}[label={list:first},caption=SC\_Exit.]
                            interrupt->Halt();
                            break;
        \end{lstlisting}
    En ce qui concerne la récupération de la valeur \textit{main}, elle peut être récupérée en lisant le registre 4 après avoir modifié le fichier \textit{start.S} pour enregistrer le registre 2 dans le registre 4 au lieu du registre 0, de la façon suivante : 
        
        \begin{lstlisting}[label={list:first},caption=\_\_start at start.S]
                .globl __start
                .ent    __start
            __start:
                jal main
                move    $4,$2       
                jal Exit     
                .end __start
        \end{lstlisting}
        
       Ainsi, la valeur du registre est enregistrée dans le registre 4, mais jusqu'à maintenant, je ne sais pas vraiment si je devrais réinscrire cette valeur dans le registre 2 comme résultat de l'appel système \textit{Exit}.
       
        
    \end{homeworkSection}
    \begin{homeworkSection}{Appel système GetChar}
        Pour la mise en oeuvre de cet appel système, il a été suffisant d'appeler la fonction \textit{SynchGetChar} et d'inscrire la valeur résultat dans le registre 2.
       
       \end{homeworkSection}
    \begin{homeworkSection}{Appel système GetString}
        
        Inspiré par l'implementation de \textit{PutString} cet appel système appelle le \textit{SynchGetString} puis écrit le résultat dans l'adresse de l'espace utilisateur employant l'appel de la fonction \textit{CopyStringToMachine} ; il faut se rappeler de toujours insérer un '\textbackslash0' à la fin.
        
          \end{homeworkSection}
    \begin{homeworkSection}{Bonus printf, PutInt, GetInt}
        
        Cette partie n'a pu être mise en oeuvre, car mon binôme a été malade. J'ai donc dû travailler seul, et n'ai pas eu le temps de tout faire. Mais je vais le faire, même seul, car mon objectif est d'apprendre au maximum. 
       
    \end{homeworkSection}

    \begin{homeworkSection}{tests}
        Pour le test, j'ai mis en oeuvre quelques fonctions supplémenaires dans \textit{putchar.c}.
        Afin de tester les appels système \textit{GetString} et \textit{PutString}, j'ai créé une boucle qui travaille comme le \textit{consoleTest} sous \textit{userprog} qui lit une chaîne de caractères et la reproduit, ce qui offre la possibilité de tester différentes longueurs de chaîne pour les deux appels système. 
        
    \end{homeworkSection}
\end{homeworkProblem}
%----------------------------------------------------------------------------------------
\end{document}


